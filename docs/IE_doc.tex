\documentclass{article}

\usepackage{latexsym}
\usepackage{fancyhdr}
\usepackage{amsmath}
\usepackage{hyperref}
\usepackage{listings}
\usepackage{pdfpages}
\usepackage[margin=1.2in]{geometry}
\lhead{Charles Pehlivanian}
\chead{October 15, 2016}
\rhead{Implied Engine Documentation [Brief]}
\cfoot{\thepage}
\pagestyle{fancy}

\hypersetup{colorlinks=true, urlcolor=blue}

\begin{document}

\title{Implied Future Pricing Engine}
\author{Charles Pehlivanian}

\maketitle

\begin{abstract}

We present the results of an application of the Implied Engine logic to a hypothtical market closely resembling CME WTI Crude (CL) futures, ICE/IPE Brent futures (B), or any one of the crude product markets. The algorithm as implemented here produces implied quote prices and sizes using only user-enteredquote information in leg, calendar spread, and strategy (strips, packs, etc.) markets. The algorithm is modular, and each leg quote (size, price) can be computed on its own execution thread. We exploit this to test a distributed version of the code, using a typical consumer-producer queue with worker tasks consuming a user quote and generating multiple downstream implied quotes per input. 
\end{abstract}

\section*{Workflow}
A high-level workflow is presented. The application subscribes to the user-only (`\"non-implied'\') quote feed for the respective exchange, and computes both top-of-book user-only and implied quotes for each leg. We present here a simple case in which 3 legs products are considered, and user quotes from the legs and 6 calendar spread products are consumed.

\includegraphics[scale=.44]{/home/charles/ClionProjects/Implied_Price_Engine_All/docs/workflow.pdf}

Notes
\begin{enumerate}
\item The initial multiplexor separates the flow of 3 leg and 6 spread products and forwards them to one of 9 quote publishers.
\item The main algorithm and most of the work is performed in the oval nodes. Here multple leg, calendar spread quotes are consumed and an implied quote (price,size) is computed for the specified leg. There are 3 Impl{\_}Quote{\_}Sub nodes in this case. Care must be taken to ensure that these passes are threadsafe.
\item Separate user and implied quote objects (uQuote, iQuote) are maintained, and best-of or merged quotes are computed in a threadsafe fashion.
\end{enumerate}

\section*{Implementation}
The main entry point is the ImpliedServer class template, consisting of 3 components: a Client object for reading of quote information via socket, an ImpliedEngine<N> object which is the computation engine, and a threadpool object for asynchronous calculations off of the main calc thread. Appropriate serialization and barriers are implemented within the ImpliedEngine<N> class to ensure thread safety. The interface can be found on \href{https://github.com/pehlivanian/Implied{\_}Price{\_}Engine}{\it https://github.com/pehlivanian/Implied{\_}Price{\_}Engine/}

\begin{verbatim}
template<int N>
class ImpliedServer
{
public:
    ImpliedServer(bool process_feed=true) :
      p_(std::make_unique<impl<ImpliedServer<N>>>()) { init_(); }
    void process() { preload_tasks_(); profiled_process_tasks_(); };
 
    //...
private:
    void preload_tasks_();
    void process_tasks_();

    void init_();
    std::unique_ptr<impl<ImpliedServer>> p_;
};

template<int N>
struct impl<ImpliedServer<N>>
{
    impl() :
            IE_(std::make_unique<ImpliedEngine<N>>()),
            C_(std::make_unique<Client>(8008, (char*)"0.0.0.0")),
            pool_(std::make_unique<threadpool>()) 
   {}

    std::unique_ptr<ImpliedEngine<N>> IE_;
    std::unique_ptr<Client> C_;
    std::unique_ptr<threadpool> pool_;

};

\end{verbatim}

We use the Eigen library for our linear algebra engine, and custom data structures, custom allocators, fast forward{\_}list, fast binary heap, fast, customized  graph search/traversal algorithms for efficiency.

Within the ImpliedEngine<N> class, the main calculation algorithm makes use of some notions from linear algebra and graph theory. A typical graph for a 12-leg problem is shown, without explanation.

\includegraphics[scale=1.25]{/home/charles/ClionProjects/Implied_Price_Engine_All/docs/test_case.pdf}
\section*{Benchmarks}

The theoretical complexity of the algorithm is $O(n^2 log(n))$, where $n$ is the number of legs to calculate. I don't believe we can do any better than this, for theoretical reasons, unless we constrain the problem based on ad-hoc market structure considerations (no liquidity in certain markets, so we eliminate them from the calculation, etc.). In any case, the algorithm is highly configurable and will allow for most such constraints.

We first empirically verify the algorithmic complexity by plotting empirical (single-core, single-threaded) CPU times with a theoretical fit. The times given are for a single update step consisting of 

\begin{enumerate}
\item pubishing parsed quote update to ImpliedEngine<N> class;
\item updating user quote object for this market;
\item calculating {\it all} implied leg quotes affected by this update (usually $N$ legs);
\item merging user, implied quotes into single quote object.
\end{enumerate}

The qualitative fit is good, with constant $C=.1223$. The fit was performed using the scipy orthogonal directions regression library.

\includegraphics[scale=.75]{/home/charles/ClionProjects/Implied_Price_Engine_All/docs/algo_complexity.pdf}


We also can calculate the update time (time start: immediately after incoming quote is parsed, time end: full update of quote objects and task returns success).  The following tables show times for a single update step (defined above) on a 4-core CentOS linux commodity machine. Times are in microseconds. Note that this time includes an update of the user quote object with the incoming quote, a step that would have to be performed in any case whether we were computing implied prices or not.

Benchmark tests were run on a 2.3GHz Intel Xeon E5-2686 v4 Broadwell, 4-core machine with 16Gb RAM.
 
\clearpage
\begin{table}
\centering
\begin{tabular}{|l|c|c|c|c|c|}
\hline
N & Average & Min & Max & Stddev & Num \\
\hline
0 & 14.563158 & 11 & 31 & 3.567040 & 9500 \\ 
1 & 14.295474 & 11 & 30 & 3.323515 & 9500 \\ 
2 & 14.150211 & 11 & 29 & 3.369054 & 9500 \\ 
3 & 14.012737 & 11 & 28 & 2.926719 & 9500 \\ 
4 & 14.050211 & 11 & 29 & 3.124248 & 9500 \\ 
5 & 13.997158 & 11 & 28 & 3.082419 & 9500 \\ 
6 & 13.974526 & 11 & 27 & 2.821791 & 9500 \\ 
7 & 14.220842 & 11 & 29 & 3.148978 & 9500 \\ 
8 & 13.908105 & 11 & 27 & 2.767451 & 9500 \\ 
9 & 13.945158 & 11 & 27 & 2.911735 & 9500 \\ 
\hline
\end{tabular}
\caption{12-Leg Problem, 10 Iterations}
\label{tab:template}
\end{table}

\begin{table}
\centering
\begin{tabular}{|l|c|c|c|c|c|}
\hline
N & Average & Min & Max & Stddev & Num \\
\hline
0 & 11.265895 & 8 & 26 & 3.154317 & 9500 \\ 
1 & 10.876737 & 8 & 24 & 2.717391 & 9500 \\ 
2 & 10.807263 & 8 & 23 & 2.611048 & 9500 \\ 
3 & 10.714737 & 8 & 23 & 2.612919 & 9500 \\ 
4 & 10.792737 & 8 & 23 & 2.466470 & 9500 \\ 
5 & 10.839158 & 8 & 23 & 2.601639 & 9500 \\ 
6 & 10.902632 & 8 & 24 & 2.816397 & 9500 \\ 
7 & 10.886947 & 8 & 25 & 2.793006 & 9500 \\ 
8 & 10.790632 & 8 & 23 & 2.615368 & 9500 \\ 
9 & 10.914737 & 8 & 24 & 2.721616 & 9500 \\ 
\hline
\end{tabular}
\caption{11-Leg Problem, 10 Iterations}
\label{tab:template}
\end{table}

\begin{table}
\centering
\begin{tabular}{|l|c|c|c|c|c|}
\hline
N & Average & Min & Max & Stddev & Num \\
\hline
0 & 10.218000 & 8 & 25 & 2.835815 & 9500 \\ 
1 & 9.975474 & 8 & 23 & 2.480480 & 9500 \\ 
2 & 9.948947 & 8 & 24 & 2.475615 & 9500 \\ 
3 & 9.834000 & 8 & 22 & 2.142498 & 9500 \\ 
4 & 10.079263 & 8 & 24 & 2.683880 & 9500 \\ 
5 & 9.976737 & 8 & 24 & 2.604584 & 9500 \\ 
6 & 10.039368 & 8 & 22 & 2.475212 & 9500 \\ 
7 & 9.976737 & 8 & 22 & 2.221967 & 9500 \\ 
8 & 9.970211 & 8 & 23 & 2.303404 & 9500 \\ 
9 & 9.810105 & 8 & 22 & 2.097212 & 9500 \\ 
\hline
\end{tabular}
\caption{10-Leg Problem, 10 Iterations}
\label{tab:template}
\end{table}

\begin{table}
\centering
\begin{tabular}{|l|c|c|c|c|c|}
\hline
N & Average & Min & Max & Stddev & Num \\
\hline
0 & 8.095368 & 6 & 20 & 1.988481 & 9500 \\ 
1 & 8.238316 & 6 & 18 & 2.100899 & 9500 \\ 
2 & 8.033263 & 6 & 20 & 1.907523 & 9500 \\ 
3 & 8.079474 & 6 & 21 & 2.086936 & 9500 \\ 
4 & 7.972211 & 6 & 17 & 1.614646 & 9500 \\ 
5 & 7.932211 & 6 & 18 & 1.664525 & 9500 \\ 
6 & 8.004211 & 6 & 20 & 1.844572 & 9500 \\ 
7 & 8.142632 & 6 & 21 & 2.089525 & 9500 \\ 
8 & 8.077263 & 6 & 21 & 2.061618 & 9500 \\ 
9 & 8.084421 & 6 & 20 & 2.041992 & 9500 \\ 
\hline
\end{tabular}
\caption{9-Leg Problem, 10 Iterations}
\label{tab:template}
\end{table}

\begin{table}
\centering
\begin{tabular}{|l|c|c|c|c|c|}
\hline
N & Average & Min & Max & Stddev & Num \\
\hline
0 & 6.198421 & 5 & 15 & 1.519532 & 9500 \\ 
1 & 6.165895 & 5 & 15 & 1.528084 & 9500 \\ 
2 & 6.145053 & 5 & 17 & 1.666477 & 9500 \\ 
3 & 6.159158 & 5 & 15 & 1.467598 & 9500 \\ 
4 & 6.061684 & 5 & 15 & 1.484377 & 9500 \\ 
5 & 6.272000 & 5 & 17 & 1.800889 & 9500 \\ 
6 & 6.078842 & 5 & 16 & 1.576353 & 9500 \\ 
7 & 6.072632 & 5 & 16 & 1.454141 & 9500 \\ 
8 & 6.256316 & 5 & 15 & 1.591008 & 9500 \\ 
9 & 6.163158 & 5 & 15 & 1.569364 & 9500 \\
\hline
\end{tabular}
\caption{8-Leg Problem, 10 Iterations}
\label{tab:template}
\end{table}

\begin{table}
\centering
\begin{tabular}{|l|c|c|c|c|c|}
\hline
N & Average & Min & Max & Stddev & Num \\
\hline
0 & 4.422737 & 3 & 12 & 1.067384 & 9500 \\ 
1 & 4.415579 & 3 & 12 & 1.096525 & 9500 \\ 
2 & 4.411895 & 3 & 10 & 0.936406 & 9500 \\ 
3 & 4.397053 & 3 & 12 & 1.036156 & 9500 \\ 
4 & 4.316632 & 3 & 12 & 0.947155 & 9500 \\ 
5 & 4.356000 & 3 & 12 & 1.034819 & 9500 \\ 
6 & 4.353895 & 3 & 12 & 0.960325 & 9500 \\ 
7 & 4.664421 & 3 & 13 & 1.448348 & 9500 \\ 
8 & 4.301895 & 3 & 12 & 0.919325 & 9500 \\ 
9 & 4.434211 & 3 & 12 & 1.131518 & 9500 \\
\hline
\end{tabular}
\caption{7-Leg Problem, 10 Iterations}
\label{tab:template}
\end{table}

\begin{table}
\centering
\begin{tabular}{|l|c|c|c|c|c|}
\hline
N & Average & Min & Max & Stddev & Num \\
\hline
0 & 3.025474 & 2 & 7 & 0.816572 & 9500 \\ 
1 & 3.023053 & 2 & 7 & 0.795552 & 9500 \\ 
2 & 2.967789 & 2 & 7 & 0.788788 & 9500 \\ 
3 & 3.022316 & 2 & 7 & 0.803408 & 9500 \\ 
4 & 3.050737 & 2 & 7 & 0.852258 & 9500 \\ 
5 & 2.963895 & 2 & 7 & 0.751642 & 9500 \\ 
6 & 2.983474 & 2 & 7 & 0.791071 & 9500 \\ 
7 & 3.108421 & 2 & 8 & 0.966451 & 9500 \\ 
8 & 3.275158 & 2 & 8 & 1.100045 & 9500 \\ 
9 & 2.990526 & 2 & 7 & 0.799854 & 9500 \\
\hline
\end{tabular}
\caption{6-Leg Problem, 10 Iterations}
\label{tab:template}
\end{table}

\clearpage
\section*{Curve Evolution}

This is a typical snapshot of the user vs. implied price (no size) for a simulated data set. The simulator produces prices in legs and calendar spread contracts, maintaining no-arbitrage conditions across the curve, and producing realistic bid-ask spreads and shape throughout the curve. This snapshot represents the market at sequence number 1350.


The implied quote curve gives a more comprehensive, arguably better notion of fair value across the curve. Note that implied prices are narrower starting at about the second or third leg, matching experience in these markets.


A full evolution for a randomly selected interval of sequence numbers is given in the file {\it quote{\_}evolution.pdf}.

\includegraphics[scale=.75]{/home/charles/ClionProjects/Implied_Price_Engine_All/docs/quote_comp.pdf}


\end{document}
